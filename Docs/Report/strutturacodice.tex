\chapter{Struttura codice}

Il codice si trova tutto nella cartella "./Src/" (soruces). Suddiviso in tre sottocartelle "./Server", "./Client" e "./Common". Le dipendenze sono le seguenti:
\\
\\
"boggle\_server" ("paroliere\_srv") eseguibile dipende da "boggle\_server.c", "server.c", "common.c".
\\
"boggle\_client" ("paroliere\_cl") eseguibile dipende da "boggle\_client.c", "client.c", "common.c".
\\
\\
"boggle\_server.c", "server.c" dipendono da "server.h".
\\
"boggle\_client.c", "client.c" dipendono da "client.h".
\\
\\
"server.h", "client.h" dipendono da "common.h".
\\
\\
"common.c" dipende da "common.h".
\\
\\
Si noti che "common.h" e "common.c" includono librerie, dichiarazioni ed implementazioni comuni a server e client. "boggle\_server.c" e "boggle\_client.c" contengono solamente i setup e lo scheletro del loro funzionamento. L'effettiva complessa e sostanziale implementazione \'e demandata rispettivamente ai "server.c" e "client.c" files. "server.h" contiene le dichiarazioni di funzioni e le strutture dati condivise tra "boggle\_server.c" e "server.c". Similmente per il client.
\\
\\
A seguire un diagramma (o grafo) delle dipendenze per il server. Quello del client \'e analogo.


% Grafo dipendenze file.
\usetikzlibrary{shapes,arrows,positioning}

% Setup.
\begin{tikzpicture}[
    node distance = 2cm and 3cm,
    file/.style = {rectangle, draw, rounded corners, minimum width=3cm, minimum height=2cm, align=center},
    arrow/.style = {->, thick}
]

% Primo livello.
\node[file] (boggle_server) {boggle\_server};

% Secondo livello.
\node[file, below left=of boggle_server] (boggle_server_c) {boggle\_server.c};
\node[file, below=of boggle_server] (server_c) {server.c};
\node[file, below right=of boggle_server] (common_c) {common.c};

% Terzo livello.
\node[file, below=of server_c] (server_h) {server.h};

% Quarto livello.
\node[file, below=of server_h] (common_h) {common.h};

% Connessioni.
\draw[arrow] (boggle_server) -- (boggle_server_c);
\draw[arrow] (boggle_server) -- (server_c);
\draw[arrow] (boggle_server) -- (common_c);
\draw[arrow] (server_c) -- (server_h);
\draw[arrow] (boggle_server_c) -- (server_h);
\draw[arrow] (server_h) -- (common_h);
\draw[arrow] (common_c) -- (common_h);

\end{tikzpicture}
% Fine grafo dipendenze file.




