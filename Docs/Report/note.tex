
\chapter*{Note}
\label{ch:note}

Si consiglia la lettura del codice in caso di necessit\'a, in quanto \'e stato molto commentato, forse anche pedantemente (pi\'u del necessario) e quindi potrebbe risultare pi\'u chiaro di tale relazione riassuntiva. Il codice \'e stato scritto in inglese per essere compreso potenzialmente da un pubblico pi\'u ampio e pubblicato su \hyperlink{https://github.com/JuliusNixi/Boggle}{GitHub}.

\vspace*{5mm}

\epigraph{"Scrivete programmi che facciano una cosa e che la facciano bene. Scrivete programmi che funzionino insieme. Scrivete programmi che gestiscano flussi di testo, perch\'e quella \'e un'interfaccia universale."}{\textit{Peter H. Salus, A Quarter Century of Unix, Unix philosophy. }}


