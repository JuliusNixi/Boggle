% Asterisco per nascondere numerazione.
\chapter*{Note}

Si consiglia la lettura del codice in caso di necessit\'a, \'e stato molto commentato, forse pi\'u del necessario e potrebbe risultare pi\'u chiaro di tale relazione riassuntiva. Il codice \'e scritto in inglese per essere compreso potenzialmente da un pubblico pi\'u ampio e pubblicato su {\color{cyan}\href{https://github.com/JuliusNixi/Boggle}{GitHub}}. Per ottemperanza alle richieste, le specifiche scritte nel testo del progetto sono tutte soddisfatte anche in italiano, ad esempio, il client accetta anche i comandi in italiano (oltre a quelli in inglese), ed entrambe gli eseguibili gestiscono gli argomenti passati da riga di comando italiani (oltre a quelli in inglese). Il codice \'e stato sviluppato su macOS (Sonoma 14.5) (Darwin Kernel Version 23.5.0) (arm64) (sfruttando le chiamate POISX) e testato sulla macchina di laboratorio Linux (laboratorio2.di.unipi.it).

\vspace{5mm}

\epigraph{"Scrivete programmi che facciano una cosa e che la facciano bene. Scrivete programmi che funzionino insieme. Scrivete programmi che gestiscano flussi di testo, perch\'e quella \'e un'interfaccia universale."}{\textit{Peter H. Salus, A Quarter Century of Unix, Unix philosophy. }}

