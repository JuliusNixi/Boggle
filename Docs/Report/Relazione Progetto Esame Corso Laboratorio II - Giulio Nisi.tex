% Sviluppato con Texifier per macOS.
% Se si utilizza il suo generatore automatico LaTeX TexpadTeX prestare attenzione perch� alcuni pacchetti non funzionano correttamente, ad esempio hyperref.
% Una volta generato, se su Texifier, se si sta usando il generatore automatico LaTeX TexpadTeX, premere un tasto nel codice per generare l'indice, che altrimenti talvolta non appare automaticamente anche senza errori.
% Per evitare tutte queste seccature si pu� utilizzare (molto meglio e consgiliato) il generatore manuale LaTeX pdfLaTeX, sempre supportato da Texifier.

% Carattere dimensione 12.
\documentclass[12pt]{report}

% Per la stampa fronte-retro sostituire con:
% \documentclass[12pt, twoside]{report}

% Margini (4cm a sx, 2.5cm a dx, 2.5cm in alto, 2.5cm in basso).
\usepackage[top=2.5cm, bottom=2.5cm, left=4cm, right=2.5cm, centering]{geometry}

% Per la stampa fronte-retro sostituire con: 
% \usepackage[top=2.5cm, bottom=2.5cm, inner=4cm, outer=4cm, right=2.5cm, centering]{geometry}

% Interlinea.
\linespread{1.5}

% Librerie utili.
% Applicazione regole di scrittura per la lingua italiana.
\usepackage[italian]{babel}
% Codifica UTF-8. 
\usepackage[utf8]{inputenc}
% Matematica.
\usepackage{amsmath} 
% Links.
\usepackage{hyperref}
\hypersetup{   
    pdftitle={Relazione Progetto Esame Corso Laboratorio II - Giulio Nisi},
    pdfpagemode=FullScreen,
}
% Citazione.
\usepackage{epigraph}
% Grafo dipendenze files.
\usepackage{tikz} 
% Colora link GitHub.
\usepackage{xcolor}
% Inserimento di immagini.
\usepackage{graphicx} 
% Posizionamento delle immagini.
\usepackage{float} 

% Stili pagina per il frontespizio e numerazione pagine.
\usepackage{scrlayer-scrpage} 
\ifoot[]{}
\cfoot[]{}
\ofoot[\pagemark]{\pagemark}
\pagestyle{scrplain}
% Font Times New Roman (simile).
\usepackage{mathptmx} 
% Per la formattazione dei titoli delle sezioni, capitoli, etc.
\usepackage{titlesec} 
% Formato delle intestazioni.
\titleformat{\chapter}[block]{\normalfont\LARGE\bfseries}{\thechapter.}{0.5em}{\LARGE}
\titlespacing*{\chapter}{0pt}{-20pt}{25pt}

\begin{document}

% Frontespizio.
\begin{titlepage}
\begin{figure}
    \centering\includegraphics[scale=0.5]{cherubino-logo-unipi.png}
\end{figure}

\begin{center}
    {\LARGE{ Corso di Laurea in Informatica \\}}
    \vspace{2cm}
    {\Large { PROGETTO ESAME CORSO LABORATORIO II }}\\
    \vspace{2cm}
    {\Large { "Il Paroliere" videogioco online scritto in C, multi-threaded, CLI, POSIX, client/server }}
\end{center}

\vspace{2cm} 

\begin{minipage}[t]{0.47\textwidth}
	{\large{\bf Professori:\\ Patrizio Dazi, Luca Ferrucci}}
	\vspace{0.5cm}

\end{minipage}\begin{minipage}[t]{0.47\textwidth}\raggedleft
	{\large{\bf Candidato: \\ Giulio Nisi\\ }}
\end{minipage}

\vspace{25mm}

\centering{\large{\bf ANNO ACCADEMICO 2023/2024 }}
\end{titlepage}
% Fine frontespizio.

% Indice e numerazione pagine.
\tableofcontents
\thispagestyle{empty}
\newpage
\addtocontents{toc}{\protect\thispagestyle{empty}}
\setcounter{page}{1}
% Fine indice.

% Inizio contenuto documento.

% Note.
% Per nasconderlo dalla numerazione nell'indice.
\addcontentsline{toc}{chapter}{Note} 
% File note.

\chapter*{Note}
\label{ch:note}

Si consiglia la lettura del codice in caso di necessit\'a, in quanto \'e stato molto commentato, forse anche pedantemente (pi\'u del necessario) e quindi potrebbe risultare pi\'u chiaro di tale relazione riassuntiva. Il codice \'e stato scritto in inglese per essere compreso potenzialmente da un pubblico pi\'u ampio e pubblicato su \hyperlink{https://github.com/JuliusNixi/Boggle}{GitHub}.

\vspace*{5mm}

\epigraph{"Scrivete programmi che facciano una cosa e che la facciano bene. Scrivete programmi che funzionino insieme. Scrivete programmi che gestiscano flussi di testo, perch\'e quella \'e un'interfaccia universale."}{\textit{Peter H. Salus, A Quarter Century of Unix, Unix philosophy. }}



% Fine note.

\chapter{Strutture dati ed algoritmi}

\section{Server}

\subsection{Lista giocatori}

La struttura dati principale usata per la gestione dei clients/giocatori, \'e una lista concatenata di elementi con una struttura e puntatore al prossimo elemento della lista. \'E stata scelta per:
 \\ - Possibilit\'a di gestire potenzialmente infiniti giocatori, risorse computazionali permettendo.
 \\ - Allocazione dinamica on demand che permette di sprecare poca memoria (non preallocandola) quando i giocatori si scollegano (la memoria viene liberata), o sono assenti.
 \\ - Il vantaggio che tutte le operazioni riguardanti un solo determinato client possono essere svolte dal relativo thread attraverso il puntatore alla struttura rappresentante suddetto giocatore e qui si hanno tutte le informazioni che potrebbero servire.
  \\
 Lo svantaggio sostanziale \'e che talvolta dobbiamo scorrere tutta la lista sincronizzandoci opportunamente con particolare attenzione, perch\'e, in caso contrario, a differenza di un array per esempio, potrebbero verificarsi problemi molto pi\'u gravi in certe fasi, come nell'aggiunta o rimozioni di elementi. Per far ci\'o si utilizza (ma non \'e l'unico) listmutex.
 \\
 Tale lista \'e condivisa tra clients registrati e non.
 \\
 
\subsection{Matrice di gioco}

La matrice di gioco, avendo dimensione predeterminata alla compilazione e costante \'e allocata staticamente globalmente come matrice di caratteri (char[][]). Il 'Qu' viene salvato come 'Q', ma gestito opportunamente.
\\

\subsection{Parole e dizionario}

Le parole del file dizionario sono memorizzate in un array (words) di stringhe (char**) allocato dinamicamente, sia l'array, che le stringhe stesse. Le parole sono caricate tutte in memoria, leggendo il file, solo una volta inizialmente, limitando l'impatto delle costose operazioni di I/O con il disco. Vi \'e poi una copia di words (words\_valid) allocata in memoria, dinamicamente, per\'o, NON riallocando le stringhe ed utilizzando inutilmente altra memoria, ma semplicemente copiando i puntatori da words. Ad ogni inizio nuovo gioco, contestualmente al cambio della matrice di gioco, i puntatori alle stringhe di words\_valid vengono aggiornati, quelli delle parole presenti nella nuova matrice, rimangono invariati (copiati da words), mentre quelli delle parole NON presenti, vengono incrementati fino a raggiungere il terminatore (\textbackslash0) di stringa. Vi \'e poi un'altra copia di words\_valid (words\_validated), allocata identicamente, per ogni giocatore registrato, il cui puntatore si trova nella struttura rappresentate tale giocatore. Quando una parola valida (presente in words\_valid E in words\_validated) viene sottomessa da un utente registrato, per la PRIMA volta, words\_validated viene aggiornato, incrementandone il puntatore fino al terminatore, cos\'icch\'e ad nuova sua sottomissione della medesima parola, essa possa essere rifiutata senza attribuirne nuovamente il punteggio al player.
\\
Riassumendo, le parole del file dizionario vengono tutte cercate ad ogni cambio di matrice di gioco (inizio nuovo gioco) (ma NON lette dal file, vengono caricate tutte solo la prima volta in words) nella matrice stessa per aggiornare words\_validated, del quale ogni giocatore (registrato) avr\'a una copia personale per tracciare quelle gi\'a sottomesse. 
\\

\subsection{Algoritmo di ricerca}

Per gestire la ricerca delle parole proposte si \'e scelto di utilizzare il seguente algoritmo. Per ogni parola viene visitata tutta la matrice e per ogni suo carattere (della matrice) invocata la funzione searchWordInMatrix(), la quale parte dal carattere della matrice specificato (con indici i di riga e j di colonna della corrente iterazione) e ricorsivamente controlla che i caratteri adiacenti [(i, j+1),(i+1, j),(i,j-1),(i-1,j)] nella matrice siano corrispondenti al prossimo carattere di parola cercato. Se non vi \'e corrispondenza, o i o j sono fuori dal range di grandezza della matrice, si ritorna un fallimento, altrimenti, se il prossimo carattere di parola voluto \'e identico al terminatore di stringa, allora significa che precedentemente abbiamo trovato tutte le lettere disposte adiacentemente e quindi ritorniamo un successo. Bisogna prestare attenzione a "marcare" i caratteri gi\'a visitanti per evitare di poter utilizzare pi\'u volte lo stesso carattere della matrice nella composizione di una parola o di perdersi in loop infiniti. Per quanto riguarda la complessit\'a? Vediamo un esempio. Per semplicit\'a assumiamo di avere M una matrice N x N (utilizzando sempre matrici quadrate), di avere K parole sul file dizionario, e di avere una parola W, da cercare, in input, ossia verificarne la presenza nella lista delle parole E nella matrice. Con la predetta soluzione, dopo aver letto dal file le parole ed averle inserite in words, si dovr\'a riempire words\_valid (solo una volta ad ogni inizio gioco), scorreremo quindi words in K passi, ad ogni passo iteriamo su M effettuando $N \cdot N$ passi, invocando ad ognuno, searchWordInMatrix(), la quale per trovare la parola al massimo effettuer\'a proprio altri $N \cdot N$ passi che rappresentano la parola di lunghezza massima che si pu\'o trovare nella matrice, vediamo a seguire un esempio con N = 5 dove con la numerazione \'e riportato solamente uno dei possibili percorsi completi che searchWordInMatrix() potr\'a intraprendere per trovare\\\\ la parola di lunghezza massima 25:
$
\begin{bmatrix}
1 & 2 & 3 & 4 & 5\\
10 & 9 & 8 & 7 & 6\\
11 & 12 & 13 & 14 & 15\\
20 & 19 & 18 & 17 & 16\\
21 & 22 & 23 & 24 & 25\\
\end{bmatrix}
$.
\\
\\
Quindi ricapitolando $ K \cdot N \cdot N \cdot N \cdot N = N^4K $ la pima volta, a seguire, ogni parola cercata da un giocatore comporter\'a solamente la ricerca in words\_valid (ed il controllo in words\_validated, ma questo si far\'a ad accesso diretto se trovata, dato che gli array sono allineati) con costo $ K $. In conclusione credo sia una discreta implementazione, che sfrutta la potenza dell'aritmetica dei puntatori, le stringhe sono effettivamente in memoria solo una volta e a fronte di un costo iniziale pi\'u elevato permette di rispondere pi\'u velocemente a seguire ai giocatori.
\\

\subsection{Coda di fine gioco}

Per la struttura del progetto vista, la coda richiesta dalle specifiche, non sarebbe stata necessaria, anzi \'e risultata un'inutile complicazione. Sarebbe stato sufficiente sbloccare i threads dei clients e segnalarli di inviare il messaggio di fine gioco con la scoreboard al client gestito. Ogni thread cos\'i, sfruttando la peculiaret\'a sopra spiegata, avrebbe potuto inviare il messaggio utilizzando il puntatore alla struttura del giocatore in possesso che fornisce gi\'a tutte le informazioni necessarie. Comunque per aderenza alle richieste, la coda \'e stata utilizzata. Essa \'e stata implementata come lista concatenata, ogni elemento \'e una struttura contenente un puntatore al corrispettivo client, un puntatore ad un messaggio (con struttura come richiesta) contenente nel campo data, il nome del client (se registrato, un placeholder altrimenti) ed il punteggio ottenuto nel gioco separati con una virgola, ed infine un puntatore all'elemento successivo. Per la sincronizzazione viene utilizzato (ma non \'e l'unico) queuemutex.
	
\section{Client}

Nel client vengono utilizzate due liste allocate dinamicamente. La prima di una struttura contenente stringhe allocate dinamicamente, dove ogni stringa ha una lunghezza massima BUFFER\_SIZE. La seconda di una struttura contenente messaggi che rappresentano le risposte ricevute dal server. Per sapere il peculiare motivo di queste fare riferimento alle pagine seguenti. Nessun particolare algoritmo \'e utilizzato.





\chapter{Struttura codice}{\label{ch:strutturacodice}}
Il codice si trova tutto nella carte Src\ (soruces).
\\
Le dipendenze sono le seguenti:
\\
\\
boggle\_server (paroliere\_srv) eseguibile dipende da boggle\_server.c, server.c, common.c..
\\
boggle\_client (paroliere\_cl) eseguibile dipende da boggle\_client.c client.c common.c..
\\
\\
boggle\_server.c, server.c dipendono da server.h..
\\
boggle\_client.c, client.c dipendono da client.h..
\\
\\
server.h, client.h dipendono da common.h..
\\
\\
common.c dipende da common.h..
\\
\\
Si noti che common.h e common.c includono dichiarazioni ed implementazioni comuni a server e client.
\\
\\
A seguire un diagramma delle dipendenze per il server. Quello del client \'e analogo.





\usetikzlibrary{shapes,arrows,positioning}


\begin{tikzpicture}[
    node distance = 2cm and 3cm,
    file/.style = {rectangle, draw, rounded corners, minimum width=3cm, minimum height=2cm, align=center},
    arrow/.style = {->, thick}
]

% First layer
\node[file] (boggle_server) {boggle\_server};

% Second layer
\node[file, below left=of boggle_server] (boggle_server_c) {boggle\_server.c};
\node[file, below=of boggle_server] (server_c) {server.c};
\node[file, below right=of boggle_server] (common_c) {common.c};

% Third layer
\node[file, below=of server_c] (server_h) {server.h};

% Fourth layer
\node[file, below=of server_h] (common_h) {common.h};

% Connections
\draw[arrow] (boggle_server) -- (boggle_server_c);
\draw[arrow] (boggle_server) -- (server_c);
\draw[arrow] (boggle_server) -- (common_c);
\draw[arrow] (server_c) -- (server_h);
\draw[arrow] (boggle_server_c) -- (server_h);
\draw[arrow] (server_h) -- (common_h);
\draw[arrow] (common_c) -- (common_h);

\end{tikzpicture}






\chapter{Struttura programmi}

\section{Server e client in comune}

\subsection{Funzioni comuni}

Il server ed il client condividono librerie utilizzate e molto codice per delle funzionalit\'a richieste comuni grazie ai files common.h e common.c. 
\\
Enunciamo sinteticamente le funzioni condivise. Abbiamo una funzione per fare il parsing dell'IP ottenuto dalla riga di comando, una per trasformare le stringhe in UPPERCASE o lowercase, le fondamentali receiveMessage() e sendMessage() che servono a scambiarsi i messaggi tra server e client, una funzione destroyMessage() per deallocare tali messaggi, due wrapper mLock() e mULock(), rispettivamente delle funzioni pthread\_mutex\_lock() e pthread\_mutex\_unlock(), utilizzati per controllare gli errori, una funzione handleError() dedicata alla gestione degli errori, una funzione wrapper printff() di quella della libreria, printf(), per stampare messaggi su stdout ed errori su stderr, che insieme ad un mutexprint permette la stampa di stringhe multiple senza che ne vengano interposte altre da altri thread quando necessario, una funzione createBanner() che crea delle stringhe carine da stampare per suddividere l'output e facilitarne la lettura ed infine una funzione threadSetup() che si occupa di associare un distruttore ad ogni thread, richiamato alla sua terminazione. Tali funzioni sono identiche in server e client.
 
 \subsection{Funzioni comuni ma con differente implementazione}

 I thread condividono il possedere un thread dedicato alla gestione dei segnali ricevuti. L'implementazione del thread differisce leggermente tra i due, poich\'e il server aveva la necessit\'a di gestire il SIGALRM in modo impegnativo (perch\'e usato per terminare il gioco), che il client non doveva. I server ed il client hanno due versioni differenti della funzione threadDestructor() che come predetto viene chiamata alla terminazione di qualunque thread, ed in base al chiamante esegue dei cleaner differenti. Per concludere vi \'e una funzione atExit() che viene chiamata al termine del programma dal thread main che si occupa anch'essa di fare pulizia.
 
\section{Gestione segnali}

Come anticipato, vi \'e un thread dedicato alla gestione dei segnali. Questo intercetta i segnali ricevuti da gestire, attraverso un loop, con una sigwait(). Inizialmente lo sviluppo \'e stato fatto con la registrazione dei signals handlers, ma questo esponeva ad innumerevoli difficolt\'a dovute al fatto che il segnale pu\'o finire ad un thread casuale e sopratutto che nel suo gestore, si \'e costretti ad utilizzare un numero estremamente ridotto di funzioni che devono avere la caratteristica di essere async-signal-safe. Con l'uso di un thread dedicato e della sigwait() invece sappiamo gi\'a il thread che ricever\'a il segnale e possiamo gestirlo liberamente.

\section{Server}

\subsection{Thread main}

Il server, dopo un'iniziale fase di setup, con inizializzazioni di varie strutture dati, registrazione del thread distruttore e della funzione atExit(), controllo degli argomenti ricevuti da riga di comando, avvio e configurazione del thread per la gestione dei segnali, registrazione del gestore dell'unico segnale (SIGUSR1) non gestito dal predetto thread, caricamento del file dizionario e matrici (se presente), apre un socket, si mette in ascolto, avvia il primo gioco (impostando un timer di fine partita con la chiamata alarm()) e si mette ad accettare clients indefinitivamente in un ciclo. All'arrivo di un client (connessione), lo accetta, inizializza un nuovo elemento della lista clients, aggiungendolo a tale lista (con dovute sincronizzazioni) ed infine avvia un thread (clientHandler()) dedicato alla gestione del giocatore, passandogli come argomento il puntatore alla propria struttura dati aggiunta alla lista.

 \subsection{Thread clientHandler()}

 Il neoavviato thread si mette in un loop ad attendere la ricezione di un messaggio dal client attraverso la funzione receiveMessage() che al suo interno utilizza la chiamata read() per leggere le varie componenti del messaggio dal socket utente. All'arrivo di un messaggio COMPLETO tenta di acquisire il proprio mutex. Ogni client, infatti, all'interno della propria struttura dati, ha anche un suo personale mutex, che acquisisce durante l'elaborazione di un messaggio ricevuto, per poi rilasciarlo al termine. Tale mutex \'e posto DOPO la lettura del messaggio, ma PRIMA della sua elaborazione.
 
 \subsection{Thread signalsThread()}

Quando il tempo di gioco termina, si riceve un segnale SIGALRM che verr\'a intercettato dal thread gestore segnali ed inizier\'a la fase pi\'u complessa del progetto. Sincronizzandosi opportunamente bloccher\'a tutti i threads clientHandler() acquisendo ciascun loro mutex. Se non dovesse riuscire ad acquisirne uno, ritenter\'a. In questo modo si \'e scelto di dare la priorit\'a al thread clientHandler() che sta gestendo un messaggio, questo ci garantisce la piacevole propriet\'a che tutte le richieste ricevute PRIMA dello scadere del timer siano processate come \'e giusto che sia. A seguire, il thread signalsThread() abilita la pausa, modificando una variabile globale (pauseon), che in seguito istruir\'a i clientHandler() threads di rispondere alle richieste di conseguenza (siamo in pausa). Ad ognuno di questi threads da parte del signalsHandler() che sta gestendo il SIGALRM verr\'a inviato il segnale SIGUSR1 per il quale nel main era stato registrato un handler (questo segnale NON \'e gestito dal signalThread(), ma ogni clientHandler() thread lo ricever\'a e gestir\'a). Lo scopo di questa azione \'e interrompere eventuali read() sulle quali il clientHandler() si dovesse trovare, perch\'e adesso (dopo esser stati sbloccati dal signalsHandler() con il rilascio del proprio mutex) i clientHandler() threads dovranno riempire la coda con i messaggi contenenti il punteggio del giocatore con le dovute sincronizzazioni (usando il queuemutex). Quando la coda sar\'a stata riempita, il signalsThread() thread avvier\'a il thread scorer() che stiler\'a la classifica finale recuperando i messaggi dalla coda ed ordinandoli per punteggio discendente con una chiamata qsort(). Al termine del scorer() thread, il signalsThread() che lo avr\'a aspettato con una join, ribloccher\'a tutti i clientHandler() threads (sempre acquisendo ciascun loro mutex scorrendo la lista), li comunicher\'a che la scoreboard CSV \'e pronta e loro (i clientHandler()) dopo esser stati tutti liberati, invieranno la scoreboard ai propri (corrispettivi) utenti. A questo punto signalsThread() non dovr\'a far altro che avviare il thread gamePause() che eseguir\'a la sleep di durata della pausa, signalsThread() far\'a il join su questo thread. Nel frattempo ovviamente tutti i clientHandler() saranno liberi di operare rispondendo liberamente alle richieste dei giocatori (tenendo conto che la pausa \'e stata abilitata). Al termine di gamePause() thread, signalsThread() bloccher\' nuovamente tutti i clientHandler() threads, disabiliter\'a la pausa, imposter\'a una nuova matrice (dal file o causale), aggiorner\'a il words\_validated array cercando tutte le parole del dizionario nella nuova matrice, imposter\'a di conseguenza il words\_validated di ciascun giocatore ed infine avvier\'a un nuovo gioco (nuova chiamata ad alarm()) e si metter\'a in attesa di nuovi segnali, cos\'i il ciclo potr\'a ricominciare.

\section{Client}

Lo sviluppo del client \'e stato sorprendentemente pi\'u arduo di quello del server, nonostante la sua abbastanza oggettiva semplicit\'a. Il concetto chiave su cui mi sono concentrato \'e stato la pulizia della stampa delle risposte del server e la successiva stampa (e gestione degli input) del prompt. Pu\'o sembrare superficiale, ma senza la giusta attenzione, l'interfaccia grafica risultava disastrosa. Pu\'0 sembrare un dettaglio, ma \'e stato difficile far fronte a questa questa esigenza,  sulla quale mi sono incaponito nel non voler scendere a compromessi. Nello specifico, il problema e\' che quando viene ricevuta e stampata una risposta dal server, l'utente potrebbe essere nel mezzo di una digitazione e aver gi\'a inserito dei caratteri nel buffer STDIN. Per vedere il problema si pu\'o guardare il codice nel file brokeninputoutput.c, nella cartella Studies/. Ho individuato due modi per affrontare il problema:
\\
- Prevenirlo. Sviluppando il client come un singolo thread, che attende che l'utente completi l'input (premendo ENTER, che tramite '\textbackslash n' interrompe la read()), poi esegue l'input ed infine controlla se ci sono risposte del server da stampare e ricomincia il ciclo. Oppure, sviluppare il client come multithread e sincronizzare il thread che gestisce l'input con quello che stampa le risposte del server. Entrambe le soluzioni, hanno lo stesso difetto: la read() deve essere obbligatoriamente bloccante, per essere sicuri di leggere l'intero buffer STDIN.
\\
- Risolverlo, lasciando che il buffer STDIN possa "sporcarsi" e pulirlo quando necessario. Sembra semplice, ma non lo \'e, ho trovato solo metodi non funzionanti su internet.
\\
Ho finito per tentare con una nota libreria chiamata ncurses. Questa permette di gestire il terminale nel dettaglio, ma era inappropiata. Risolveva il problema, ma generava una complessit\'a insostenibile per le semplici stampe che dovevo fare. 
\\
Dopo innumerevoli tentativi sono riuscito a risolvere il problema in modo semplice, ricorrendo a due librerie ma senza doverne fare un uso estensivo. Ho utilizzato termios, vedere termiostest.h nella cartella Studies/ e fcntl. Utilizzo due threads, uno per la gestione dell'input utente e della stampa delle risposte del server (quest'ultime vengono lette da una lista) ed uno che legge le risposte del server dal socket e le inserisce in lista. Interrompo la read() nel primo thread molto spesso e controllo se ci sono risposte del server, in caso affermativo, le stampo, ripulisco lo STDIN con una getchar() NON BLOCCANTE e ritorno al prompt pulito, in modo che se l'utente avesse digitato qualcosa di incompleto possa riscriverlo e/o modificarlo (a seguito della risposta). Invece, se non ci sono risposte dal server, la read() riprende con lo STDIN inserito dall'utente che non si accorge di nulla. L'unico svantaggio \'e che le stampe delle risposte del server non avvengono in modo asincrono, si ottiene un effetto simile per l'utente, perch\'e la read() viene interrotta ogni pochi millisecondi per stampare le risposte del server. La ricezione delle risposte e l'inserimento in lista invece e\' eseguita da un thread ad-hoc e quindi e\' totalmente asincrono, ma la stampa no. La sincronizzazione tra i due threads avviene con un mutex. Le due librerie citate servono solo a rendere la getchar() non bloccante e permettermi quindi di pulire l'STDIN. Per approfondire vedasi inputoutputasynctests.c nella cartella Studies/.
\\
Un ulteriore problema affrontato \'e stato quello di non sapere anticipatamente la grandezza dell'input dell'utente. Non vi sono infatti limitazioni sulla lunghezza della parola o del nome utente, per poter gestire input di arbitraria dimensione si leggono BUFFER\_SIZE caratteri dallo STDIN con una read(), essi vengono inseriti in un char array[BUFFER\_SIZE] statico, ma da qui vengono poi copiati in una stringa allocata sullo heap dinamicamente (sempre di lunghezza BUFFER\_SIZE) che viene aggiunta ad una lista concatenata di stringhe. Scegliendo cos\'i un BUFFER\_SIZE adeguato allocheremo sempre la dimensione di memoria che pi\'u si avvicina alla grandezza dell'input dell'utente evitando di preallocare tanta memoria inutilmente che poi comunque se allocata staticamente potrebbe terminare (in caso di input utente grandi), con una lista di stringhe si ovvia a questo problema.




\chapter{Struttura tests}

Inizialmente l'idea era quella di scrivere i tests in C in un file eseguibile separato. Ed ho intrapreso questa strada, testimone il fatto che nella cartella Tests/ ho lasciato il file tests.c, il quale funziona, si pu\'o compilare ed eseguire con make. Tuttavia con lo sviluppo mi sono reso conto che per come avevo strutturato il progetto era disastroso. Avrei potuto testare poche singole funzioni ma non ad un livello soddisfacente nell'insieme. Causa il fatto di avere i codici divisi in pi\'u file ed di aver utilizzato per molte funzionalit\'a variabili globali condivise. Allora ho guardato qualche framework di testing in C, ma non ho trovato nulla che facesse al caso mio. Cos\'i messo alle strette ho pensato di approcciare il problema in modo "particolare". Ho pensato che avrei potuto utilizzare la semplicit\'a e la potenza di Python che conosco un poco, per fare i tests in modo esterno, ma qui simulando proprio il comportamento di utenti e server anche su larga scala. E ha funzionato benissimo! Solamente \textbf{mi dispiace di esser dovuto andare a scomodare un linguaggio che esula dal corso}. Tuttavia in questo modo ho potuto eseguire moltissimi tests che sono stati vitali e mi hanno permesso di correggere innumerevoli bugs. Per eseguire questi tests \'e sufficiente:
\\
1. Avviare il server.
\\
2. Entrare nella cartella Tests/ ed eseguire "python3 pythontests.py server\_ip porta\_server".
\\
Ci sono alcuni parametri all'inizio del file che possono essere modificati se voluto.
\\
Il file crea tantissimi processi client e per ognuno di essi svolge un numero di azioni che possono essere esattamente tutte quelle che un vero utente potrebbe eseguire. Inviare comandi non validi, registrarsi, richiedere la matrice, sottomettere una parola, di quelle valide e non valide, uscire e persino disconnettersi forzatamente. Tutto viene svolto casualmente con un intensivo uso di random (ma comunque replicabile tramite seed). Tra ogni azione si aspetta un numero casuale di millisecondi. I comandi vengono sottomessi da python, creando il processo del client, collegandosi tramite PIPE al suo stdin. L'output, ossia le risposte del client ricevute dal server viene salvato in dei file di logs in Tests/logs/, insieme all'elenco dei comandi inviati. 
\\
I tests dovrebbero funzionare con Python 3 con le librerie installate di default senza richiedere alcuna risorsa esterna.


\chapter{Compilazione ed esecuzione}

Il codice pu\'o essere compilato dalla root del progetto.
\\
\\
Valgono i seguenti targets main:

\begin{itemize}
\item make = Compila tutto: client e server.
\item make execs = Compila ed esegue il server, con gli argomenti di default (si veda il Makefile).
\item make execc = Compila ed esegue il client, con gli argomenti di default (si veda il Makefile).
\item make clean = Rimuove tutti i files oggetto, eseguibili, files di logs dei tests Python e termina forzatamente tutti gli eventuali processi in esecuzione di server e client.
\end{itemize}
\leavevmode 
Ovviamente oltre a questi nel Makefile \'e scritto tutto l'albero delle dipendenze.
\\
L'utilizzo degli eseguibili si pu\'o svolgere facendo riferimento al seguente formato:
\\
Server Usage: paroliere\_srv server\_ip server\_port [--matrices matrices\_filepath] [--duration game\_duration\_in\_minutes] [--seed rnd\_seed] [--dic dictionary\_filepath].
\\
Client Usage: paroliere\_cl server\_ip server\_port.
\\
\\
\textbf{ATTENZIONE:} L'utilizzo di alcuni percorsi assoluti hard-coded (ad esempio il percorso del file dizionario di default quando non specificato da CLI) assume che la \textbf{WORKING DIRECTORY} sia la \textbf{ROOT} del progetto. Stessa cosa per lo script "./Tests/Python/pythontests.py".


\chapter{Miscellanea}

Le funzionalit\'a di cui sono pi\'u orgoglioso sono:
\\
1. Il server pu\'o far giocare potenzialmente infiniti giocatori.
\\
2. La matrice di gioco pu\'o essere N x N di dimensione arbitraria.
\\
3. Le parole possono avere lunghezza arbitraria.
\\
4. L'input dell'utente nel client pu\'o essere di lunghezza arbitraria.
\\
5. Le stampe nel client sono perfette, la GUI non mostra mai stampe incoerenti.
\\
6. I tests eseguiti mostrano una buona resa di server e clients.
\\
7. Il client e\' semplicissimo poich\'e il grosso del lavoro lo esegue \'e il server.
\\

La sincronizzazione \'e stata volutamente fatta solamente tramite mutex, dove talvolta le variabili di condizione o i semafori si sarebbero potuti utilizzare, per cercare di rendere il codice pi\'u semplice ed intuitivo possibile, seppur abbia ancora della complessit\'a.
\\
Quanti giocatori riesce a gestire nella pratica? Ci sono criticit\'a? Dai tests effettuati purtroppo si, nonostante come detto al punto 1, in teoria i giocatori potrebbero essere infiniti, vi \'e un problema. Quando si arriva all'incirca a 500 giocatori (attivi che fanno richieste), ma gi\'a dai 400, sulla macchina di laboratorio, un po' di pi\'u sul mio pc, il server diventa lentissimo nel rispondere ed accettare nuovi utenti paralizzando praticamente il gioco, questo a causa del fatto che sincronizzare 500 threads, considerando anche che in alcuni punti viene fatta attesa attiva (di cui mi sono pentito). Credo che il problema potrebbe essere parzialmente risolto aumentando di parecchio la priorit\'a (rispetto ai clientHandler() thread) dei thread signalsThread(), che gestisce i segnali ed il gioco, e quella del thread main che accetta i nuovi clients, ci ho provato molto, ma alla fine mi sono arreso perch\'e non riesco a farlo con le funzioni di scheduling dei pthreads.
 \\
 \\
 Le parole, indipendentemente da come arrivino dal client e da come siano scritte nei file dizionario vengono convertite in UPPERCASE dal server. Anche i caratteri componenti la matrice di gioco sono gestiti esclusivamente in UPPERCASE, sia nella generazione casuale, sia nella lettura da file (se scritte in lowercase in quest'ultimo vengono convertite in UPPERCASE). Gli unici caratteri ammessi per le parole, le matrici e i nickname degli utenti sono: \#define ALPHABET "abdcdefghijklmnopqrstuvxyz".
 \\
 L'input del client, invece, viene tutto convertito a lowercase prima di essere elaborato ed inviato, quindi il comando inserito "MaTrIcE" sarà valido come "matrice".
 
  

% Fine contenuto documento.

\end{document}

