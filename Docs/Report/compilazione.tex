\chapter{Compilazione ed esecuzione}

Il codice pu\'o essere compilato dalla root del progetto.
\\
\\
Valgono i seguenti targets (phony) che sono utili shortcut:

\begin{itemize}
\item make = Compila tutto: client, server e tests.
\item make execs = Compila ed esegue il server, con gli argomenti di default (si veda il "./Makefile").
\item make execc = Compila ed esegue il client, con gli argomenti di default (si veda il "./Makefile").
\item make tests = Compila ed esegue i tests, con gli argomenti di default (si veda il "./Makefile").
\item make clean = Rimuove tutti i files oggetto, files eseguibili, files di logs dei tests e termina forzatamente tutti gli eventuali processi in esecuzione di server, client e tests.
\end{itemize}
\leavevmode 
Ovviamente oltre a questi nel "./Makefile" \'e scritto tutto l'albero delle dipendenze ed \'e quindi possibile invocare "make" con il target file desiderato.
\\
L'utilizzo degli eseguibili si pu\'o svolgere facendo riferimento al seguente formato:
\\
Server Usage: ./paroliere\_srv server\_ip server\_port [--matrices matrices\_filepath] [--duration game\_duration\_in\_minutes] [--seed rnd\_seed] [--dic dictionary\_filepath].
\\
Client Usage: ./paroliere\_cl server\_ip server\_port.
\\
Tests Usage: ./paroliere\_tests server\_ip server\_port.
\\
\\
\textbf{ATTENZIONE:} L'utilizzo di alcuni percorsi assoluti hard-coded (ad esempio il percorso del file dizionario di default quando non specificato da CLI) assume che la \textbf{WORKING DIRECTORY} sia la \textbf{ROOT} del progetto. Quindi NON FARE MAI cose del tipo "cd Bin/" e "./qualcosa".