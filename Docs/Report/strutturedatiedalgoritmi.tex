\chapter{Strutture dati ed algoritmi}

\section{Server}

\subsection{Lista giocatori}

La struttura dati adottata per la gestione dei clients/giocatori, \'e una lista concatenata di elementi con una loro struttura e puntatore al prossimo elemento della lista. \'E stata scelta per:

\begin{itemize}
\item Possibilit\'a di gestire potenzialmente infiniti giocatori, risorse computazionali permettendo.
\item Allocazione dinamica on demand che permette di sprecare poca memoria (non preallocandola) quando i giocatori si scollegano (la memoria viene liberata), o sono assenti.
\item Tutte le operazioni riguardanti un solo determinato client possono essere svolte dal relativo thread attraverso il puntatore alla struttura rappresentante suddetto giocatore e qui si hanno tutte le sue informazioni che potrebbero servire.
\end{itemize}
\leavevmode
Lo svantaggio sostanziale \'e che talvolta dobbiamo scorrere interamente la lista sincronizzandoci opportunamente con particolare attenzione, perch\'e, in caso contrario, a differenza di un array per esempio, potrebbero verificarsi problemi pi\'u gravi in certe fasi, come nell'aggiunta o rimozione di elementi (potremmo avere dei memory leaks ad esempio). 
\\
Per sincronizzarsi si utilizza (ma non \'e l'unico) "listmutex".
\\
Tale lista \'e condivisa tra clients registrati e non.
 \\
 
\subsection{Matrice di gioco}

La matrice di gioco, avendo dimensione predeterminata alla compilazione e costante \'e allocata staticamente globalmente (in "server.h") come matrice di caratteri ("char[][]"). Il 'Qu' viene salvato come 'Q', ma gestito contandolo come singolo carattere.
\\

\subsection{Parole e dizionario}

Le parole del file dizionario sono memorizzate in un array "words" di stringhe ("char**") allocato dinamicamente, sia l'array, che le stringhe stesse. Le parole sono caricate tutte in memoria, leggendo il file, solo una volta inizialmente, limitando l'impatto delle costose operazioni di I/O con il disco. Vi \'e poi una copia di "words", "words\_valid", allocata in memoria, dinamicamente, per\'o, NON riallocando le stringhe, sprecando inutilmente altra memoria, ma semplicemente copiando i puntatori da "words". Ad ogni inizio nuovo gioco, contestualmente al cambio della matrice di gioco, i puntatori alle stringhe di "words\_valid" vengono aggiornati, quelli delle parole presenti nella nuova matrice (trovate con l'algoritmo che segue), rimangono invariati (copiati da "words"), mentre quelli delle parole NON presenti, vengono incrementati fino a raggiungere il terminatore ("\textbackslash0") di stringa. Vi \'e poi un'altra copia di "words\_valid", "words\_validated", allocata identicamente, per ogni giocatore REGISTRATO, il cui puntatore si trova nella struttura rappresentate tale giocatore. Quando una parola valida (presente in "words\_valid" E in "words\_validated") viene sottomessa da un utente registrato, per la PRIMA volta in questo gioco, "words\_validated" viene aggiornato, incrementandone il puntatore fino al terminatore, cos\'icch\'e ad una nuova sua sottomissione della medesima parola nello stesso gioco, essa possa essere rifiutata senza attribuirne nuovamente il punteggio al player.
\\

\subsection{Algoritmo di ricerca}

Per gestire la ricerca delle parole si \'e scelto di utilizzare il seguente algoritmo. Ad ogni cambio della matrice di gioco, per ogni parola del file dizionario caricata in memoria, viene visitata tutta la matrice di gioco cio\'e per ogni suo carattere (della matrice) invocata la funzione "searchWordInMatrix()", la quale parte dal carattere della matrice specificato (con indici i di riga e j di colonna della corrente iterazione) e ricorsivamente controlla che i caratteri adiacenti [(i, j+1),(i+1, j),(i,j-1),(i-1,j)] nella matrice siano corrispondenti al prossimo carattere di parola cercato. Se non vi \'e corrispondenza, o i o j sono fuori dal range di grandezza della matrice, si ritorna un fallimento, altrimenti, se il prossimo carattere di parola voluto \'e identico al terminatore di stringa, allora significa che precedentemente abbiamo trovato tutte le lettere disposte adiacentemente e quindi ritorniamo un successo. Bisogna prestare attenzione a "marcare" i caratteri gi\'a visitanti per evitare di poter utilizzare pi\'u volte lo stesso carattere della matrice nella composizione di una parola o di perdersi in loop infiniti, nelle chiamate successive di funzione ricorsive. Per quanto riguarda la complessit\'a? Per semplicit\'a assumiamo di avere M, una matrice N x N (utilizzando matrici quadrate), di avere K parole sul file dizionario, e di avere una parola W, da cercare, in input, ossia verificarne la presenza nella lista delle parole E nella matrice corrente di gioco. Con la predetta soluzione, dopo aver letto dal file dizionario le parole ed averle inserite in "words", si dovr\'a riempire "words\_valid" (solo una volta ad ogni inizio gioco), scorreremo quindi "words" in K passi, ad ogni passo iteriamo su M effettuando $N \cdot N$ passi, invocando ad ognuno, "searchWordInMatrix()", la quale per trovare la parola al massimo effettuer\'a proprio altri $N \cdot N$ passi che rappresentano la parola di lunghezza massima che si pu\'o trovare nella matrice.
\\
Quindi ricapitolando paghiamo $ K \cdot N \cdot N \cdot N \cdot N = N^4K $ la prima volta (ad ogni inizio gioco), a seguire, ogni parola cercata da un giocatore comporter\'a solamente la ricerca in "words\_valid" (ed il controllo in "words\_validated", ma questo si far\'a ad accesso diretto se trovata, dato che gli array sono allineati) con costo $ K $. In conclusione, credo sia una discreta implementazione, che sfrutta la potenza dell'aritmetica dei puntatori, cio\'e le stringhe sono effettivamente in memoria solo una volta e a fronte di un costo iniziale pi\'u elevato permette di rispondere pi\'u velocemente a seguire ai giocatori.
\\

\subsection{Coda di fine gioco}

Per la struttura del progetto usata, la coda richiesta dalle specifiche, non sarebbe stata necessaria, anzi \'e risultata un'inutile complicazione. Sarebbe stato sufficiente sbloccare i threads dei clients e segnalarli di inviare il messaggio di fine gioco con la scoreboard al client gestito. Ogni thread avrebbe potuto inviare il messaggio utilizzando il puntatore alla struttura del giocatore in possesso che fornisce gi\'a tutte le sue informazioni necessarie. Comunque per aderenza alle richieste, la coda \'e stata utilizzata. Essa \'e stata implementata come lista concatenata, ogni elemento \'e una struttura contenente un puntatore al corrispettivo client, un puntatore ad un messaggio (con struttura come richiesta) contenente nel campo "data", il nome del client (se registrato, un placeholder altrimenti) ed il punteggio ottenuto nel gioco separati con una virgola, ed infine un puntatore all'elemento successivo della coda. Per la sincronizzazione viene utilizzato (ma non \'e l'unico) "queuemutex".
	
\section{Client}

Nel client vengono utilizzate due liste concatenate allocate dinamicamente. La prima di una struttura contenente stringhe, dove ogni stringa ha una lunghezza massima "BUFFER\_SIZE". La seconda di una struttura contenente messaggi che rappresentano le risposte ricevute dal server. Nessun particolare algoritmo \'e utilizzato, se non lo scorrimento delle liste.



